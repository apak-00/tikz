\documentclass[usenames, dvipsnames ]{article}
\usepackage{amsmath, amssymb, wasysym, pgfplots, tikz, graphicx, color, bbm, cmbright, mathtools}
\usepackage[margin=0.45in]{geometry}

\usetikzlibrary{calc,trees,positioning,arrows,chains,shapes.geometric,%
	decorations.pathreplacing,decorations.pathmorphing,shapes,%
	matrix,shapes.symbols, positioning,fit,calc, angles, quotes, patterns,%
	decorations.text}

\usetikzlibrary{external}
\tikzexternalize[]

\newcommand{\includetikz}[1]{%
	\tikzsetnextfilename{#1}%
	\input{#1.tex}%
}

\begin{document}
	
	\clearpage
	\section{Triangulation}
	
	\begin{figure*}[h!]
		\centering
		\includetikz{tikz/triangulation_1}
		\caption{Triangulation principle}
	\end{figure*}

	\begin{figure*}[h!]
		\centering
		\includetikz{tikz/triangulation_2}
		\caption{Triangulation Principle. Step 1 - System of Equations, Step 2 - Expand to $x,y,z$, Step 3: Get $m,m'$, Step 4: Compute $P_w$}
	\end{figure*}
	
	\begin{figure}[h!]
		\centering
		\includetikz{tikz/triangulation_reconstruction}
		\caption{$P_w$ - 3D object point, $\hat{P}_w$ - reconstructed point. That's why the previous system of equations won't work.}
	\end{figure}
	
	\clearpage
	\section{Epipolar Geometry}
	
	\begin{figure}[h!]
		\centering
		\includetikz{tikz/epipolar_1}
		\caption{Epipolar Geometry: Focal points, epipoles and epipolar lines. All epipolar lines intersect at epipole.}
	\end{figure}
	
	\begin{figure}[h!]
		\centering
		\includetikz{tikz/epipolar_2}
		\caption{Epipolar Geometry - Modeling}
	\end{figure}
	
	\begin{figure}[h!]
		\centering
		\includetikz{tikz/epipolar_3}
		\caption{Epipolar Geometry - Modeling}
	\end{figure}
	



\end{document}




